\section{Ableitung}\label{sec:ableitung}

\begin{definition}{Definition}
    Die Funktion, die jeder Stelle $x$ den Wert $f'(x)$ zuordnet, wird \emph{Ableitungsfunktion} von $f$ genannt.
    Es werden folgende Schreibweisen verwendet:
    \begin{multicols}{5}
        \begin{itemize}
            \item $f'(x)$
            \item $\frac{\d f}{\d x}$
            \item $\frac{\d y}{\d x}$
        \end{itemize}
    \end{multicols}
\end{definition}

\begin{definition}{Satz}
    Gegeben ist die Funktion $f(x) = c$ mit $c \in \R$.
    Dann gilt: $f'(x) = 0$ für jedes $x \in \R$.
\end{definition}

\textbf{Begründung:} Die Steigung einer horizontalen Geraden ist an jeder Stelle $x$ gleich $0$.

\begin{definition}{Satz}
    Gegeben ist die Funktion $f(x) = x^k$ mit $k \neq 0$.
    Dann gilt: $f'(x) = k \cdot x^{k-1}$.
\end{definition}

\begin{definition}{Ableitung höherer Ordnung}
    Die \emph{zweite Ableitung} erhält man, indem man die Ableitungsfunktion noch einmal ableitet: \[f''(x) = \frac{\d^2 f}{\d x^2} = \frac{\d}{\d x}(f'(x))\]
    Die \emph{dritte Ableitung} erhält man, indem man die zweite Ableitung noch einmal ableitet: \[f'''(x) = \frac{\d^3 f}{\d x^3} = \frac{\d}{\d x}(f''(x))\]
    Und so weiter.
\end{definition}

\textbf{Bemerkung:} Die zweite Ableitung gibt Auskunft über die Veränderung der Ableitung;
wenn man z.B.\ die Geschwindigkeit $v(t)$ ableitet, erhält man die Beschleunigung.

\subsection{Ableitungsregeln}\label{subsec:ableitungsregeln}

\begin{definition}{Faktorregel}
    $(c \cdot f)'(x) = c \cdot f'(x)$
\end{definition}

\textbf{Beispiel:} $(4x^3)' = 4 \cdot (x^3)' = 4 \cdot 3x^2 = 12x^2$

\begin{definition}{Summenregel}
    $(f + g)'(x) = f'(x) + g'(x)$
\end{definition}

\textbf{Beispiel:} $(7x^5 - 3x^3 + 5x^2 - 14x + 6)' = (7x^5)' - (3x^3)' + (5x^2)' - (14x)' + (6)' = 35x^4 - 9x^2 + 10x - 14$

\begin{definition}{Produktregel}
    $(u \cdot v)'(x) = u'(x) \cdot v(x) + u(x) \cdot v'(x)$
\end{definition}

\textbf{Beispiel:} $f(x) = (3x^3 + x^2)(4x^2 + 1)$.
Gesucht ist $f'(x)$.

$u = 3x^3 + x^2$ und $u' = 9x^2 + 2x$\\
$v = 4x^2 + 1$ und $v' = 8x$

$\Rightarrow f'(x) = u'v + uv' = (9x^2 + 2x) \cdot (4x^2 + 1) + (3x^3 + x^2) \cdot 8x = 36x^4 + 9x^2 + 8x^3 + 2x + 24x^4 + 8x^3 = 60x^4 + 16x^3 + 9x^2 + 2x$

\begin{definition}{Quotientenregel}
    $\left( \frac{u}{v} \right)'(x) = \frac{u'(x) \cdot v(x) - u(x) \cdot v'(x)}{(v(x))^2}$
\end{definition}

\begin{definition}{Kettenregel}
    $(F \circ u)'(x) = F'(u) \cdot u'(x)$ wobei
    \begin{itemize}
        \item $F(u)$ die äussere Funktion
        \item $F'(u)$ die Ableitung der äusseren Funktion nach $u$
        \item $u(x)$ die innere Funktion
        \item $u'(x)$ die Ableitung der inneren Funktion nach $x$
    \end{itemize}
    sind.
\end{definition}

\textbf{Beispiel:} $f(x) = (x^3 + 4)^{-2}$

$F(u) = u^{-2}$ und $F'(u) = -2u^{-3}$\\
$u(x) = (x^3 + 4)$ und $u'(x) = 3x^2$

$\Rightarrow f'(x) = -2u^{-3} \cdot 3x^2 = -2(x^3 + 4)^{-3} \cdot 3x^2$

\subsection{Ableitungen bestimmter Funktionen}\label{subsec:ableitungen-bestimmter-funktionen}

\begin{multicols}{2}
    \begin{itemize}
        \item $(\sin (x))' = \cos (x)$
        \item $(\cos (x))' = -\sin (x)$
        \item $(e^x)' = e^x$
        \item $(a^x)' = a^x \cdot \ln (a)$
        \item $(\ln (x))' = \frac{1}{x}$
        \item $(\log_a(x))' = \frac{1}{x \cdot \ln (a)}$
    \end{itemize}
\end{multicols}

\subsection{Differenzierbarkeit}\label{subsec:differenzierbarkeit}

Es gibt Stellen, an denen nur die Funktion, nicht aber die Ableitung definiert ist.
Zum Beispiel ist die Betragsfunktion an der Stelle $x = 0$ nicht differenzierbar.

\begin{definition}{Definition}
    Eine Funktion $f(x)$ ist an der Stelle $x_0$ \emph{differenzierbar}, wenn die linksseitige mit der rechtsseitigen Ableitung übereinstimmt.\\
    Eine Funktion $f(x)$ heisst \emph{differenzierbar} oder \emph{ableitbar}, wenn die Ableitung an jeder Stelle ihres Definitionsbereichs definiert ist.
\end{definition}