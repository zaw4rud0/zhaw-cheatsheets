\section{Rekursive Strukturen und die natürlichen Zahlen}\label{sec:rekursive-strukturen-und-die-naturlichen-zahlen}

\subsection{Struktur der natürlichen Zahlen}\label{subsec:struktur-der-naturlichen-zahlen}

Wir listen einige Grundtatsachen über die Struktur der natürlichen Zahlen $\N$ auf:
\begin{itemize}
    \item Die Zahl 0 ist eine natürliche Zahl.
    \item Jede natürliche Zahl $k$ hat genau einen Nachfolger $k + 1$.
    Der Nachfolger jeder natürlichen Zahl ist wiederum eine natürliche Zahl.
    \item Die Zahl 0 ist die einzige natürliche Zahl, die kein Nachfolger ist.
    \item Natürliche Zahlen mit gleichem Nachfolger sind gleich.
    \item Enthält die Menge $X$ die 0 und mit jeder natürlichen Zahl $k$ auch deren Nachfolger $k + 1$, so bilden die natürlichen Zahlen $\N$ eine Teilmenge von $X$.
\end{itemize}
Das letzte Axiom heisst \emph{Induktionsaxiom}, da auf ihm die Beweismethode der vollständigen Induktion beruht.

\begin{definition}{}
    Die Ordnung $\leq$ auf den natürlichen Zahlen ist durch \[x \leq y \coloneqq \exists k \in \N (x + k = y)\] gegeben.
    Wir schreiben \[x < y \coloneqq x \leq \land x \neq y\]
\end{definition}

\subsection{Die vollständige Induktion}\label{subsec:die-vollstandige-induktion}

Die Induktion ist eine sehr wichtige Beweismethode in der Mathematik und in der Informatik.
Beim klassischen Induktionsbeweis geht es darum, zu zeigen, dass eine Aussage für alle natürlichen Zahlen gilt.
Sie ist folgendermassen aufgebaut:
\begin{enumerate}
    \item \textbf{Induktionsanfang:} Zeige, dass die Aussage für $n = 1$ gilt.
    \item \textbf{Induktionshypothese:} Wir nehmen an, dass die Aussage für ein allgemeines $n \in \N$ gültig ist.
    \item \textbf{Induktionsschritt:} Zeige, dass aus der Gültigkeit der Aussage für $n$ (Induktionshypothese) die Gültigkeit der Aussage für $n + 1$ folgt.
\end{enumerate}
Eine verallgemeinerte Form der Induktion erlaubt eine stärkere Induktionshypothese, nämlich dass die Aussage gültig \emph{für alle $k \leq n$} sei.

\textbf{Beispiel (Induktion):} Beweise mit der vollständigen Induktion, dass folgendes für alle $n \in \N$ gilt: \[\sum \limits_{i=0}^n 2^i = 2^{n+1} - 1\]

\textbf{Induktionsanfang (B.C.):} n = 0 \[2^0 = 1 = 2^1 - 1\]

\textbf{Induktionshypothese (I.H.):} $\forall n \in \N$ \[\sum \limits_{i=0}^n 2^i = 2^{n+1} - 1\]

\textbf{Induktionsschritt (I.S.):} $n \rightarrow n + 1$
\begin{align*}
    \sum \limits_{i=0}^{n+1} 2^i &= \sum \limits_{i=0}^{n} 2^i + 2^{n+1} \\
    &\stackrel{\text{IH}}{=} 2^{n+1} - 1 + 2^{n+1} \\
    &= 2 \cdot 2^{n+1} - 1 \\
    &= 2^{n+2} - 1
\end{align*}

% TODO: Beispiel rekursiver Algorithmus

% TODO: Beispiel rekursive Definitionen




