\section{Zahlentheorie}\label{sec:zahlentheorie}

\begin{definition}{Menge der ganzen Zahlen}
    Analog zu unserem Vorgehen mit den natürlichen Zahlen wollen wir auch die \emph{ganzen Zahlen} informell einführen.
    Wir definieren \[\Z \coloneqq \{\dots,-2,-1,0,1,2,\dots\}\]
\end{definition}

\begin{definition}{Rechenregeln auf $\Z$}
    Für alle $r,s,z \in \Z$ gelten folgende Gleichungen:
    \begin{align*}
        -1 \cdot z &= -z \\
        -(-z) &= z \\
        -z + z &= 0 & \text{Inverse Elemente bezüglich} \  +\\
        0 \cdot z &= 0 & \text{Absorbtion}\\
        1 \cdot z &= z &\text{Neutrales Element bezüglich} \ \cdot\\
        0 + z &= z &\text{Neutrales Element bezüglich} \ +\\
        r(sz) &= (rs)z &\text{Assoziativität von} \ \cdot\\
        r + (s + z) &= (r + s) + z &\text{Assoziativität von} \ +\\
        rs &= sr &\text{Kommutativität von} \ \cdot\\
        r + s &= s + r &\text{Kommutativität von} \ +\\
        r(s + z) &= rs + rz &\text{Distributivität}\\
        rx = ry &\Rightarrow x = y \lor r = 0 &\text{Kürzbarkeit}
    \end{align*}
\end{definition}

\begin{definition}{Definitionen}
    Wir definieren die \emph{Substraktion} $-\ : \Z \times \Z \rightarrow \Z$ durch \[x - y \coloneqq x + (-y)\text{,}\]
    die \emph{Betragsfunktion} $| \cdot | : \Z \rightarrow \N$ durch \[ |z| =
    \begin{cases}
        z$ falls $z \in \N \\ -1 \cdot z$ sonst$
    \end{cases}\]
    und die Relation $\leq$ durch \[x \leq y \coloneqq \exists n \in \N \left(x + n = y\right)\text{.}\]
\end{definition}

\subsection{Teilbarkeit und der Euklidische Algorithmus}\label{subsec:teilbarkeit-und-der-euklidische-algorithmus}

\begin{definition}{Definition}
    Sind $x,y \in \Z$ ganze Zahlen, so sagen wir, dass \emph{x ein Teiler von y} ist, falls es ein $k \in \Z$ gibt mit $xk = y$.
    Wir schreiben in diesem Fall $x | y$.
    Es gilt also \[x | y \coloneqq \exists k \in Z \left( y = xk \right)\]
    Mit $T(y)$ bezeichnen wir die Menge aller natürlichen Zahlen, welche Teiler von $y$ sind, also $T(x) = \{ x \in \N \mid x | y \}$.
\end{definition}

\textbf{Bemerkung:} Die Teilbarkeitsrelation ist reflexiv und transitiv auf der Menge $\Z$, auf der Menge $\N$ ist die Teilbarkeitsrelation sogar eine Halbordnung.

\begin{definition}{Satz}
    Sind $n,m \in \N \backslash \{0\}$, dann gibt es eindeutig bestimmte Zahlen $k,r \in \N$, sodass Folgendes gilt:
    \begin{multicols}{2}
        \begin{itemize}
            \item $m = kn + r$
            \item $r < n$
        \end{itemize}
    \end{multicols}
    Wir sagen in diesem Zusammenhang, dass die Zahl $r$ der Rest von der ganzzahligen Division von $m$ durch $n$ ist.
\end{definition}

\begin{definition}{KgV und ggT}
    Seien $n,m \in Z$.
    Wir definieren das \emph{kleinste gemeinsame Vielfache von $n$ und $m$} als \[kgV(n,m) \coloneqq \min \{ k \in N \mid n|k \land m|k \}.\]
    Ist $n \neq 0$ oder $m \neq 0$, dann definieren wir den \emph{grössten gemeinsamen Teiler von $n$ und $m$} als \[ggT(n,m) \coloneqq \max \{ k \in \N \mid k|n \land k|m \}.\]
\end{definition}

\begin{definition}{Satz (Euklidischer Algorithmus)}
    Für $n,m \in \N$ mit $0 < n < m$ gilt \[ggT(n,m) = ggT(n,m-n) = ggT(m,m-n).\]
\end{definition}

\begin{mainbox}{Rezept: Euklidischer Algorithmus}
    \begin{enumerate}
        \item Grössere durch kleinere Zahl dividieren.
        \item Divisor durch Rest dividieren.
        Diesen Schritt so oft wiederholen, bis kein Rest übrig bleibt.
        \item Ergebnis aufschreiben.
    \end{enumerate}
\end{mainbox}

\textbf{Beispiel: $ggT(442, 255)$}
\vspace{-\topsep}
\begin{enumerate}
    \item $442 : 255 = 1$ Rest $187$
    \item $255 : 187 = 1$ Rest $68 \\
    187 : 68 = 2$ Rest $51 \\
    68 : 51 = 1$ Rest $17 \\
    51 : 17 = 3$ Rest $0$
    \item $ggT(442, 255) = 17$
\end{enumerate}

\begin{subbox}{}
    Zwei ganze Zahlen $x,y$ heissen \emph{teilerfremd}, wenn $ggT(x,y) = 1$ gilt.
\end{subbox}

\begin{definition}{Lemma von Bézout}
    Sind $x,y \in \Z$ mit $x,y \neq 0$, dann gibt es ganze Zahlen $a,b$, sodass \[ggT(x,y) = ax + by\] gilt.
    Die Zahlen $a$ und $b$ werden \emph{Bézout-Koeffizienten} genannt.
\end{definition}

\textbf{Beispiel:} Berechnen Sie $ggT(135, 480)$ und gleichzeitig die $x,y$ mit $ggT(135, 480) = 135x + 480y$.
Wir wenden den sogenannten \emph{erweiterten Euklidischen Algorithmus} an:
\begin{alignat*}{2}
    480 &= 3 \cdot 135 + 75 \quad \quad &&\Leftrightarrow 75 = 480 - 3 \cdot 135 \\
    135 &= 1 \cdot 75 + 60 &&\Leftrightarrow 60 = 135 - 1 \cdot 75 = 135 - (480 - 3 \cdot 135) = -480 + 4 \cdot 135 \\
    75 &= 1 \cdot 60 + 15 &&\Leftrightarrow 15 = 75 - 1 \cdot 60 = 480 - 3 \cdot 135 - (-480 + 4 \cdot 135) = 2 \cdot 480 - 7 \cdot 135 \\
    60 &= 4 \cdot 15 + 0
\end{alignat*}
Wir erhalten also $ggT(480, 135) = 15 = -7 \cdot 135 + 2 \cdot 480$, also $x = -7$ und $y = 2$.

\subsection{Primzahlen}\label{subsec:primzahlen}

\begin{definition}{Definition}
    Eine natürliche Zahl $p$ ist eine \emph{Primzahl}, wenn $|T(p)| = 2$ gilt.
    Die Menge der Primzahlen bezeichnen wir mit $\P$.
\end{definition}

\textbf{Bemerkung:} Ist $p$ eine Primzahl, dann gilt $T(p) = \{ 1,p \}$.

\begin{definition}{Lemma von Euklid}
    Für $p \in \N$ mit $p \neq 1$ gilt: \[\forall n,m \in \N \left( p|nm \Rightarrow p|n \lor p|m \right) \Leftrightarrow p \in \P.\]
    Mit anderen Worten: Primzahlen haben die Eigenschaft, dass sie mit jedem Produkt auch mindestens einen der Faktoren teilen.
    Umgekehrt ist auch jede von 1 verschiedene natürliche Zahl mit dieser Eigenschaft eine Primzahl.
\end{definition}

\begin{definition}{}
    Jede ganze Zahl $z$ mit $z \notin \{ -1,1 \}$ besitzt einen Primfaktor (einen Teiler, der eine Primzahl ist).
    Formal können wir dies als \[\forall z \in \Z \left( z \notin \{-1,1 \} \Rightarrow T(z) \cap \P \neq 0 \right)\] ausdrücken.
\end{definition}

\begin{definition}{Theorem}
    Jede natürliche Zahl grösser als 1 ist das Produkt von endlich vielen Primzahlen.
\end{definition}

\begin{definition}{Primfaktorzerlegung}
    Es sei $p_i$ jeweils die $i$-te Primzahl.
    Für jede natürliche Zahl $n > 1$ gibt es eine eindeutig bestimmte, endliche Folge $a_1,\cdots,a_k$ von natürlichen Zahlen mit $a_k \neq 0$, sodass \[n = \prod_{i=1}^k p_i^{a_i}\] gilt.
\end{definition}

\subsection{Modulare Arithmetik}\label{subsec:modulare-arithmetik}

In der modularen Arithmetik geht es darum mit Restklassen zu rechnen.
Die Grundlage der modularen Arithmetik ist die ``kongruent modulo''-Relation.

\begin{definition}{Definition}
    Es sei $n \in \N$ beliebig.
    Wir definieren eine Relation $\equiv_n$ auf $\Z$ wie folgt: \[r \equiv_n s \coloneqq n|(r-s).\]
    Gilt für $r,s \in \Z$ die Relation $r \equiv_n s$, dann sagen wir, dass $r$ gleich $s$ modulo $n$ ist und wir schreiben \emph{r = s mod n}.
\end{definition}

\textbf{Bemerkung:} Die Relation $\equiv_n$ ist für jede natürliche Zahl $n$ eine Äquivalenzrelation auf $\Z$.

\textbf{Bemerkung:} Es sei $n \in \N$ beliebig.
Für je zwei ganze Zahlen $x$ und $y$ gilt $x \equiv_n y$ genau dann, wenn $x$ und $y$ denselben Rest bei Division durch $n$ haben.

\textbf{Folgerung:} Es sei $n \in \N$ beliebig.
Jede ganze Zahl $z$ steht mit genau einer natürlichen Zahl aus $\{0,\cdots,n-1\}$ in der Relation $\equiv_n$.

\begin{definition}{Definition}
    Es sei $n \in \N$ beliebig.
    Für jede ganze Zahl $z$ bezeichnen wir mit \[[z]_n \coloneqq \{ x \in \Z \mid x \equiv_n z \}\] die Äquivalenzklasse von $z$ bezüglich der Relation $\equiv_n$ und nennen diese auch die \emph{Restklassen} von $z$.
    Abkürzend bezeichnen wir $[z]_n$ auch mit $\overline{k}$, wenn $k \in \{ 0, \cdots, n-1\}$ und $z \equiv_n k$ gilt.
\end{definition}

\textbf{Folgerung:} Es sei $n \in \N$ beliebig.
Es gilt \[[z]_n = \{ z + yn \mid y \in \Z \} = \{\cdots, z-3n, z-2n, z-n,z,z+n, z+2n, z+3n, \cdots \}.\] Damit wir mit Restklassen sinnvoll rechnen können, müssen wir uns davon überzeugen, dass die Rechenoperationen unabhängig von der Wahl von Repräsentanten sind.

\textbf{Bemerkung:} Es sei $n \in \N$ beliebig.
Für ganze Zahlen $x, x'$ und $y, y'$ gelten:
\vspace{-\topsep}
\begin{itemize}
    \item $[x] = [x'] \land [y] = [y'] \Rightarrow [x + y] = [x' + y']$
    \item $[x] = [x'] \land [y] = [y'] \Rightarrow [xy] = [x'y']$
\end{itemize}

\begin{definition}{Definition}
    Es sei $n \in \N$ beliebig.
    Die Menge aller Restklassen von $\Z$ modulo $n$ bezeichnen wir mit \[\Z/n = \{ [z]_n \mid z \in \Z \} = \{ \overline{k} \mid 0 \leq k < n - 1 \land z \equiv_n k \} = \{\overline{0}, \overline{1}, \overline{2}, \cdots, \overline{n-1} \}.\]
    Wir definieren zwei Verknüpfungen $\cdot : (\Z/n)^2 \rightarrow \Z/n$ und $+ : (\Z/n)^2 \rightarrow \Z/n$ durch die Zuordnungen \[[x]_n + [y]_n \coloneqq [x + y]_n\] und \[[x]_n \cdot [y]_n \coloneqq [xy]_n.\]
\end{definition}

% TODO: Bemerkungen

\subsubsection{Chinesischer Restsatz}

\begin{definition}{Chinesischer Restsatz}
    Es seien $n_1,\dots,n_k \in \N_{>1}$ paarweise teilerfremd und weiter $y_1,\dots,y_k \in \Z$ beliebig.
    Es gibt genau eine natürliche Zahl $x < \prod_{i=1}^k n_i$ so, dass die Lösungsmenge des Systems
    \begin{align*}
        &x \equiv_{n_1} y_1 \\
        &x \equiv_{n_2} y_2 \\
        &\vdots \\
        &x \equiv_{n_k} y_k
    \end{align*}
    der Menge $[x]_{\prod_{i=1}^k} n_i$ entspricht.
\end{definition}

\textbf{Bemerkung:} Aus dem chinesischen Restsatz folgt, dass wir, um ein System simultaner Kongruenzen zu lösen, bloss eine Lösung davon kennen müssen.
Durch sukzessive Substitution genügt es also jeweils eine Lösung von einem System mit zwei Gleichungen zu finden, um beliebige Systeme lösen zu können.

\begin{subbox}{Algorithmus}
    Wir wollen ein System simultaner Kongruenzen mit zwei Gleichungen lösen, etwa
    \begin{align*}
        &x \equiv_{n_1} y_1 \\
        &x \equiv_{n_2} y_2
    \end{align*}
    mit $n_1$ und $n_2$ teilerfremd.
    Wir gehen schrittweise wie folgt vor:
    \begin{enumerate}
        \item Durch sukzessives Teilen mit Rest erhalten wir ganze Zahlen $a,b$ mit $an_1 + bn_2 = 1$.
        \item Wir setzen $x \coloneqq y_1 bn_2 + y_2 an_1$.
    \end{enumerate}
\end{subbox}

\begin{definition}{Lemma}
    Ist $a \in \Z/n$ mit $n > 0$ invertierbar, dann ist die Funktion
    \begin{align*}
        &f : \Z/p \rightarrow \Z/p \\
        &f(x) = \overline{a} \cdot x
    \end{align*}
    surjektiv.
\end{definition}

\begin{definition}{Kleiner fermatscher Satz}
    Ist $p \in \P$ und $a$ kein Vielfaches von $p$, dann gilt \[a^{p-1} \equiv_p 1.\]
\end{definition}

\textbf{Beispiel (Chinesischer Restsatz):}
\begin{align*}
    x &\equiv_{73} 5 \\
    x &\equiv_{3} 1 \\
    x &\equiv_{11} 6
\end{align*}

Wir setzen $n_1 = 73$, $n_2 = 3$ und $n_3 = 11$, und rechnen das Produkt davon aus: $n = n_1 \cdot n_2 \cdot n_3 = 2409$.

Anschliessend rechnen wir $N_1 = n_2 \cdot n_3 = 33$, $N_2 = n_1 \cdot n_3 = 803$ und $N_3 = n_1 \cdot n_2 = 219$ aus.
\begin{align*}
[33]
    _{73}^{-1} &\stackrel{*}{=} [31]_{73} \\
    [803]_3^{-1} &= [2]_3^{-1} = [2]_3 \quad \longrightarrow \quad [2]_3 \cdot [2]_3 = [4]_3 = [1]_3 \\
    [219]_{11}^{-1} &= [10]_{11}^{-1} = [10]_{11} \quad \longrightarrow \quad [10]_{11} \cdot [10]_{11} = [100]_{11} = [1]_{11}
\end{align*}
Wir haben also \[x = 5 \cdot 33 \cdot 31 + 1 \cdot 803 \cdot 2 + 6 \cdot 219 \cdot 10 = 19861\] und damit \[[19861]_{2409} = [589]_{2409}.\]

Die gesamte Lösungsmenge lautet $\{ 589 + k \cdot 2409 : k \in \Z \}.$

\begin{alignat*}{2}
    \text{(*)}\quad &73 = 2 \cdot 33 + 7 \quad \quad &&\Rightarrow 7 = 73 - 2 \cdot 33 \\
    &33 = 4 \cdot 7 + 5 &&\Rightarrow 5 = 33 - 4 \cdot 7 = 33 - 4 \cdot (73 - 2 \cdot 33) = 9 \cdot 33 - 4 \cdot 73 \\
    &7 = 1 \cdot 5 + 2 &&\Rightarrow 2 = 7 - 5 = 73 - 2 \cdot 33 - (9 \cdot 33 - 4 \cdot 73) = 5 \cdot 73 - 11 \cdot 33 \\
    &5 = 2 \cdot 2 + 1 &&\Rightarrow 1 = 5 - 2 \cdot 2 = 9 \cdot 33 - 4 \cdot 73 - 2 \cdot (5 \cdot 73 - 11 \cdot 33) = 31 \cdot 33 - 14 \cdot 73
\end{alignat*}

\textbf{Beispiel (Multiplikative Inverse):} Welche Elemente von $\Z/12$ besitzen multiplikative Inverse?
\begin{align*}
    ggT(12,1) = 1 &\Rightarrow [1]_{12}^{-1} = [1]_{12} \\
    ggT(12,5) = 1 &\Rightarrow [5]_{12}^{-1} = [5]_{12} \\
    ggT(12,7) = 1 &\Rightarrow [7]_{12}^{-1} = [7]_{12} \\
    ggT(12,11) = 1 &\Rightarrow [11]_{12}^{-1} = [11]_{12}
\end{align*}

\textbf{Beispiel (Verknüpfungstabelle):} Skizzieren Sie die Verknüpfungstabelle von $\Z/4$ und markieren Sie die Elemente, die kein multiplikatives Inverses besitzen.

\begin{center}
    \begin{tabular}{c|cccc}
        $\cdot$ & $0$ & $1$ & $2$ & $3$ \\
        \hline
        0       & 0   & 0   & 0   & 0   \\
        1       & 0   & 1   & 2   & 3   \\
        2       & 0   & 2   & 0   & 2   \\
        3       & 0   & 3   & 2   & 1
    \end{tabular}
\end{center}

\textbf{Beispiel (Multiplikatives Inverse):} Bestimmen Sie das multiplikative Inverse von $\overline{6111}$ in $\Z/6211$.

Wir wenden den erweiterten euklidischen Algorithmus an:
\begin{alignat*}{2}
    &6211 = 1 \cdot 6111 + 100 \quad \quad &&\Leftrightarrow 100 = 6211 - 6111 \\
    &6111 = 61 \cdot 100 + 11 &&\Leftrightarrow 11 = 6111 - 61 \cdot 100 = 6111 - 61 \cdot (6211 - 6111) = 62 \cdot 6111 - 61 \cdot 6211 \\
    &100 = 9 \cdot 11 + 1 &&\Leftrightarrow 1 = 100 - 9 \cdot 11 = 6211 - 6111 - 9 \cdot (62 \cdot 6111 - 61 \cdot 6211) = 550 \cdot 6211 - 559 \cdot 6111
\end{alignat*}

Damit haben wir: \[[6111]_{6211}^{-1} = [-559]_{6211} = [5652]_{6211}\]

\textbf{Beispiel (Chinesischer Restsatz):}
\begin{align*}
    x &\equiv_{7} 5 \\
    x &\equiv_{3} 2 \\
    x &\equiv_{8} 1
\end{align*}
Zuerst überprüfen wir, dass alle Moduli teilerfremd sind (Voraussetzung für den chinesischen Restsatz): \[ggT(3,7) = ggT(3,8) = ggT(7,8) = 1\]
Dann berechnen wir: $n = 3 \cdot 7 \cdot 8 = 168$ und $N_1 = n_2 \cdot n_3 = 7 \cdot 8 = 56$, $N_2 = n_1 \cdot n_3 = 3 \cdot 8 = 24$, $N_3 = n_1 \cdot n_2 = 3 \cdot 7 = 21$
\begin{align*}
    &[N_1]_{n_1}^{-1} = [56]_3^{-1} = [2]_3^{-1} = [2]_3 \\
    &[N_2]_{n_2}^{-1} = [24]_7^{-1} = [3]_7^{-1} = [5]_7 \\
    &[N_3]_{n_3}^{-1} = [21]_8^{-1} = [5]_8^{-1} = [5]_8
\end{align*}
Damit ist \[x = 2 \cdot 56 \cdot 2 + 5 \cdot 24 \cdot 5 + 1 \cdot 21 \cdot 5 = 929\] und wir schliessen ab mit \[[929]_{168} = [89]_{168}.\]
Die Lösungsmenge lautet: $\{89 + k \cdot 168 : k \in \Z\}$