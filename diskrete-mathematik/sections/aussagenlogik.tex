\section{Syntax und Semantik der Aussagenlogik}\label{sec:syntax-und-semantik-der-aussagenlogik}

\subsection{Syntax der Aussagenlogik}\label{subsec:syntax-der-aussagenlogik}

\begin{definition}{}
    Das \emph{Alphabet des Aussagenlogik} besteht aus:
    \vspace{-\topsep}
    \begin{itemize}
        \setlength{\itemsep}{0pt}
        \setlength{\parskip}{0pt}
        \setlength{\parsep}{0pt}
        \begin{multicols}{2}
            \item Konstanten $\top$ und $\bot$.
            \item Variablen $p,q,r,s,\dots,p_0,p_1,p_2,\dots$
            \item Klammern (,)
            \item Junktoren $\neg, \land, \lor, \rightarrow$
        \end{multicols}
    \end{itemize}
    \vspace{-\topsep}
    Die Menge der Variablen bezeichnen wir mit $\V$.
\end{definition}

\begin{definition}{}
    Jede Variable und jede Konstante ist eine \emph{atomare Formel}.
    Wir bezeichnen die Menge aller atomaren Formeln mit $\mathbb{A} \coloneqq \{\bot, \top, p,q,r,s,\dots,p_0,p_1,p_2,\dots\}$.
    Die \emph{Formeln} der Aussagenlogik sind dann wie folgt:
    \begin{itemize}
        \item Alle atomaren Formeln sind Formeln.
        \item Sind $P$ und $Q$ schon Formeln, dann auch: $(P \land Q), (P \lor Q), (P \rightarrow Q)$ und $\neg P$.
    \end{itemize}
    Wir schreiben $\mathbb{F}$ für die Menge aller aussagenlogischen Formeln.
\end{definition}

\subsection{Semantik der Aussagenlogik}\label{subsec:semantik-der-aussagenlogik}

\begin{definition}{}
    Eine \emph{Belegung} ist eine Zuordnung von Variablen zu Wahrheitswerten, d.h.\ eine Funktion $B : \V \rightarrow \{\true, \false\}$.
    Es sei eine Belegung $B$ gegeben.
    Die Funktion $\widehat{B}$ ist die Funktion, die jeder aussagenlogischen Formel ihren Wahrheitswert bezüglich der Belegung $B$ zuordnet, d.h.\ die Funktion $\widehat{B}: \F \rightarrow \{\false, \true\}$ ist gegeben durch:
    \begin{itemize}
        \item $\widehat{B}(\perp) = \false$ und $\widehat{B}(\top) = \true$

        \item Für beliebige Variablen $v$ gilt $\widehat{B}(v) = B(v)$

        \item Für beliebige Formeln $F$ und $G$ gilt
        $$\widehat{B}(F \land G) = \begin{cases}
                                       \true$ falls $\widehat{B}(F) = \true$ und $\widehat{B}(G) = \true \\ \false$ sonst. $
        \end{cases}$$

        \item Für beliebige Formeln $F$ und $G$ gilt
        $$\widehat{B}(F \lor G) = \begin{cases}
                                      \true$ falls $\widehat{B}(F) = \true$ oder $\widehat{B}(G) = \true \\ \false$ sonst. $
        \end{cases}$$

        \item Für beliebige Formeln $F$ gilt
        $$\widehat{B}(\neg F) = \begin{cases}
                                    \true$ falls $\widehat{B}(F) = \false \\ \false$ sonst. $
        \end{cases}$$

        \item Für beliebige Formeln $F$ und $G$ gilt $\widehat{B}(F \rightarrow G) = \widehat{B}(\neg F \lor G)$.
    \end{itemize}
\end{definition}

Die Junktoren können wir auch als boolesche Funktionen anschauen:
\begin{align*}
    \texttt{or} (x,y) &= \begin{cases}
                             \true$ falls $x = \true$ oder $y = \true \\ \false$ sonst. $
    \end{cases}\\
    \texttt{and} (x,y) &= \begin{cases}
                              \true$ falls $x = \true$ und $y = \true \\ \false$ sonst. $
    \end{cases}\\
    \texttt{not} (x) &= \begin{cases}
                            \true$ falls $x = \false \\ \false$ sonst. $
    \end{cases}
\end{align*}

\vspace{-\topsep}

Dadurch können wir die obige Definition kürzer und knapper formulieren:
\begin{multicols}{3}
    \begin{itemize}
        \item $\widehat{B}(F \land G) = \texttt{and}(\widehat{B}(F), \widehat{B}(G))$
        \item $\widehat{B}(F \lor G) = \texttt{or}(\widehat{B}(F), \widehat{B}(G))$
        \item $\widehat{B}(\neg F) = \texttt{not}(\widehat{B}(F))$
    \end{itemize}
\end{multicols}

\begin{definition}{Wahrheitstabellen}
    In einer \emph{Wahrheitstabelle einer Formel $F$} entspricht jede Spalte einer Teilformel von $F$ und jede Zeile einer Belegung der in $F$ vorkommenden Variablen.
    Es gelten folgende Bedingungen:
    \begin{itemize}
        \item In der Spalte einer Formel steht in jeder der folgenden Zeilen der Wahrheitswert dieser Formel unter der Zeile entsprechenden Belegung.
        \item Steht in einer Spalte eine Formel, dann kommen alle echten Teilformeln dieser Formeln in Spalten weiter links vor.
        \item Der letzte Eintrag der ersten Zeile ist die Formel $F$.
    \end{itemize}
\end{definition}

\textbf{Bemerkung:} Für eine bessere Übersicht werden in Wahrheitstabellen anstelle der Wahrheitswerte $\true$ und $\false$ oft 1 und 0 verwendet.

\begin{multicols}{4}
    \centering
    \begin{tabular}{|c|c||c|}
        \hline
        $F$ & $G$ & $F \land G$ \\
        \hline
        0   & 0   & 0           \\
        0   & 1   & 0           \\
        1   & 0   & 0           \\
        1   & 1   & 1           \\
        \hline
    \end{tabular}
    \centering
    \begin{tabular}{|c|c||c|}
        \hline
        $F$ & $G$ & $F \lor G$ \\
        \hline
        0   & 0   & 0          \\
        0   & 1   & 1          \\
        1   & 0   & 1          \\
        1   & 1   & 1          \\
        \hline
    \end{tabular}
    \centering
    \begin{tabular}{|c|c||c|}
        \hline
        $F$ & $G$ & $F \rightarrow G$ \\
        \hline
        0   & 0   & 1                 \\
        0   & 1   & 1                 \\
        1   & 0   & 0                 \\
        1   & 1   & 1                 \\
        \hline
    \end{tabular}
    \centering
    \begin{tabular}{|c||c|}
        \hline
        $F$ & $\neg F$ \\
        \hline
        0   & 1        \\
        1   & 0        \\
        \hline
    \end{tabular}
\end{multicols}

\vspace{-\topsep}

\begin{definition}{Semantische Eigenschaften}
    Eine aussagenlogische Formel $A$ heisst
    \begin{itemize}
        \item \emph{Gültig} oder \emph{wahr} unter einer Belegung $B$, falls $\widehat{B}(A) = \true$.
        \item \emph{Allgemeingültig}, wenn sie unter jeder Belegung gültig ist (Tautologie).
        \item \emph{Widerlegbar}, wenn es mindestens eine Belegung gibt, unter der $A$ nicht gültig ist.
        \item \emph{Erfüllbar}, wenn es mindestens eine Belegung gibt, unter der $A$ gültig ist.
        \item \emph{Unerfüllbar}, wenn $A$ nicht erfüllbar ist.
    \end{itemize}
\end{definition}

Die obigen Begriffe können auch anhand von Wahrheitstabellen verstanden werden.
Eine aussagenlogische Formel $A$ ist
\vspace{-\topsep}
\begin{itemize}
    \item Allgemeingültig, wenn in der Wahrheitstabelle von $A$ in der letzten Spalte all Einträge $\true$ sind.
    \item Erfüllbar, wenn in der Wahrheitstabelle von $A$ in der letzten Spalte mindestens einer der Einträge $\true$ ist.
    \item Unerfüllbar, wenn in der Wahrheitstabelle von $A$ in der letzten Spalte alle Einträge $\false$ sind.
    \item Widerlegbar, wenn in einer Wahrheitstabelle von $A$ in der letzten Spalte mindestens einer der Einträge $\false$ ist.
\end{itemize}

\begin{definition}{}
    Es seien $F$ und $G$ beliebige aussagenlogische Formeln.
    Wir sagen
    \begin{itemize}
        \item \emph{F ist eine Konsequenz von G}, falls $F$ unter jeder Belegung $\true$ ist unter der $G$ $\true$ ist.
        \item $F$ und $G$ sind \emph{logisch äquivalent}, wenn $G$ und $F$ unter jeder Belegung denselben Wahrheitswert annehmen.
    \end{itemize}
    Sind $F$ und $G$ äquivalente Formeln, dann schreiben wir $F \equiv G$.
\end{definition}

\begin{definition}{}
    Eine aussagenlogische Formel ist:
    \begin{itemize}
        \item In \emph{Negationsnormalform} (NNF), wenn alle Negationen in Literalen vorkommen und wenn keine Implikationen ($\rightarrow$) vorkommen.
        \item In \emph{disjunktiver Normalform} (DNF), wenn sie mit Literalen $L_{i,j}$ von der Form ist: \[(L_{1,1} \land L_{1,2} \land \dots) \lor (L_{2,1} \land L_{2,2} \land ...) \lor (L_{3,1} \land L_{3,2} \land ...)\dots\]
        \item In \emph{konjunktiver Normalform} (KNF), wenn sie mit Literalen $L_{i,j}$ von der Form ist: \[(L_{1,1} \lor L_{1,2} \lor \dots) \land (L_{2,1} \lor L_{2,2} \lor ...) \land (L_{3,1} \lor L_{3,2} \lor ...)\dots\]
    \end{itemize}
\end{definition}

\textbf{Beispiel:} Es sei die Formel $F = p \rightarrow (q \rightarrow \neg q).$
\begin{enumerate}
    \item Ist die Formel erfüllbar? \\
    Ja, es gibt mind.\ eine Belegung, unter der die Formel gültig ist, nämlich wenn p oder q falsch ist.
    \item Ist die Formel allgemeingültig? \\
    Nein, die Formel ist nicht gültig, wenn sowohl p als auch q wahr sind.
    \item Bringen Sie die Formel $F$ in $KNF$.
    \begin{align*}
        p \rightarrow (q \rightarrow \neg q) &\equiv \neg p \lor (q \rightarrow \neg q) \\
        &\equiv \neg p \lor (\neg q \lor \neg q) \\
        &\equiv \neg p \lor \neg q
    \end{align*}
\end{enumerate}

\textbf{Beispiele ($KNF$ \& $DNF$):}
\begin{itemize}
    \item $p$ ist in $KNF$ und $DNF$.
    \item $(p \land (\neg q \land p_1)) \equiv p \land \neg q \land p_1 \equiv (p \land \neg q \neg p_1)$ ist in $KNF$ und $DNF$.
    \item $p \lor (q \rightarrow p) \equiv p \lor (\neg q \lor p)$ ist nicht in $KNF$ und nicht in $DNF$.
    \item $p \lor (\neg p \land (p \lor q))$ ist weder $KNF$ noch in $DNF$.
    \item $(p \lor q) \land (p \lor (p \lor p))$ ist in $KNF$, aber nicht in $DNF$.
\end{itemize}

\textbf{Beispiele ($KNF$ \& $DNF$):} Bringen Sie die folgenden Formeln in $DNF$ und $KNF$.
\begin{multicols}{2}
    \begin{align*}
        p \rightarrow (q \lor (p_1 \land p_2)) &\equiv \neg p \lor \underbrace{(q \lor (p_1 \land p_2))}_{DNF} \\
        &\equiv \neg p \lor ((q \lor p_1) \land (q \lor p_2)) \\
        &\equiv \underbrace{(\neg p \lor q \lor p_1) \land (\neg p \lor q \lor p_2)}_{KNF}
    \end{align*}
    \begin{align*}
        p \rightarrow (q \rightarrow p_1) &\equiv \neg p \lor (q \rightarrow p_1) \\
        &\equiv \underbrace{\neg p \lor \neg q \lor p_1}_{KNF \ \text{und} \ DNF}
    \end{align*}
    \begin{align*}
        (p \rightarrow q) \rightarrow p_1 &\equiv \neg (p \rightarrow q) \lor p_1 \\
        &\equiv \neg (\neg p \lor q) \lor p_1 \\
        &\equiv \underbrace{(p \land \neg q) \lor p_1}_{DNF} \\
        &\equiv \underbrace{(p \lor p_1) \land (\neg q \lor p_1)}_{KNF}
    \end{align*}
\end{multicols}