\section{Audio-Codierung}\label{sec:audio-codierung}

\begin{definition}{}
    \textbf{Abtasttheorem von Shannon:} $F_{\text{abtast}} > 2 \cdot f_{\max}$
\end{definition}

\subsection{Pulse Code Modulation (PCM)}\label{subsec:pulse-code-modulation}
\begin{itemize}
    \item \textbf{Filterung:} Zu hohe und zu tiefe Frequenzen werden entfernt, sodass der hörbare Frequenzbereich (etwa 20--20kkHz) übrig bleibt.
    \item \textbf{Abtastung:} Ein zeitkontinuierliches Signal wird in ein zeitdiskretes Signal umgewandelt, indem es in regelmäßigen Abständen registriert wird.
    \item \textbf{Quantisierung:} Die Anzahl Bit, die für die Quantisierung verwendet werden, bestimmt die Anzahl Stufen, welche für die Messung der Amplitude zur Verfügung stehen.
    Die gemessenen Werte werden der nächsten Stufe gerundet zugeordnet.
    Dabei entsteht ein Fehler, der sogenannte \emph{Quantisierungsfehler}.
    \item \textbf{Codierung:} Jedem quantisierten Messwert wird ein Wert von einer bestimmten Bitlänge zugeordnet.
    Aufgrund der Abtastrate entsteht somit ein Bitstrom, der berechnet werden kann mit: \[\text{Anzahl Bit pro Messwert} \cdot \text{Abtastfrequenz}.\]
\end{itemize}

\subsection{Wave File Format}\label{subsec:wave-file-format}

Das \emph{wave}-Dateiformat ist ein Containerformat zur digitalen Speicherung von Audiodaten, wobei der Header Informationen über die Audiodaten enthält.
Wave-Dateien enthalten normalerweise keine komprimierten Audiodaten, sondern die PCM-Rohdaten.
Die Qualität des aufgezeichneten Klangs hängt dann von zwei Werten ab, der Abtastrate (Anzahl Abtastungen pro Zeiteinheit) und der Auflösung der Quantisierung (Bit-Tiefe).

Der File Header besteht aus 44 Byte mit verschiedenen Informationen, wie Format, Anzahl Kanäle, Samples pro Sekunde, Bit pro Sample und Länge der Rohdaten.
Anschliessend folgen die digitalen Audio-Daten, jeweils abwechselnd pro Kanal.

\subsection{Verlustfreie Audio-Codierung}\label{subsec:verlustfreie-audio-codierung}

Im Gegensatz zu verlustbehaftete Audiodatenkompressionsverfahren wie MP3 oder Ogg Vorbis ist die Komprimierung bei FLAC verlustfrei, es gibt also keine Qualitätseinbusse.
Dafür sind die komprimierte Dateien aber auch um ein Vielfaches grösser.

\subsection{Verlustbehaftete Audio-Codierung}\label{subsec:verlustbehaftete-audio-codierung}

\begin{definition}{Schalldruckpegel}
    Um verlustbehaftete Codierung durchführen zu können, braucht es den \emph{Schalldruckpegel}, der die Stärke eines Schallereignisses beschreibt und folgendermassen berechnet wird:
    \[L_P = 10 \log_{10} \left( \frac{\tilde{p}^2}{p_0^2} \right)) \ \text{dB} = 20 \log_{10} \left( \frac{\tilde{p}}{p_0} \right) \ \text{dB},\] wobei $\tilde{p}$ der effektive Schalldruck in Pa ist und $p_0$ der Schalldruckpegel bei einer Schallstärke von 1kHz ist (international auf $0.00002$ Pa gesetzt).
\end{definition}
Eine Verdopplung des effektiven Schalldrucks $p$ entspricht einer Zunahme des Schallpegels um rund \textbf{6dB}, weil \[20 \cdot \log_{10}(2) = 6.02 \ \text{dB}\]
