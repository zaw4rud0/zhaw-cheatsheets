\section{Zahlensysteme}\label{sec:zahlensysteme}

\begin{comment}
    \subsection{Vergleich}\label{subsec:vergleich}
    \begin{center}
        \begin{tabular}{|c|c|c|}
            \hline
            10-er System & 2-er System & 16-er System \\
            \hline
            \hline
            0            & 0000        & 0            \\
            \hline
            1            & 0001        & 1            \\
            \hline
            2            & 0010        & 2            \\
            \hline
            3            & 0011        & 3            \\
            \hline
            4            & 0100        & 4            \\
            \hline
            5            & 0101        & 5            \\
            \hline
            6            & 0110        & 6            \\
            \hline
            7            & 0111        & 7            \\
            \hline
            8            & 1000        & 8            \\
            \hline
            9            & 1001        & 9            \\
            \hline
            10           & 1010        & A            \\
            \hline
            11           & 1011        & B            \\
            \hline
            12           & 1100        & C            \\
            \hline
            13           & 1101        & D            \\
            \hline
            14           & 1110        & E            \\
            \hline
            15           & 1111        & F            \\
            \hline
        \end{tabular}
    \end{center}
\end{comment}

\subsection{Umrechnung}\label{subsec:umrechnung}

\textbf{10-er ins 16-er System:} In einem Beispiel wollen 100 ins 16-er System umwandeln
\begin{align*}
    100_d \div 16 &= 6 \text{ Rest } 4 \\
    6 \div 16 &= 0 \text{ Rest } 6
\end{align*}
Es folgt: $100_d = 64_h$

Überprüfung: $64_h = 6 \cdot 16_d^1 + 4 \cdot 16_d^0 = 100_d$

\textbf{10-er ins 2-er System:} In diesem Beispiel wollen sehen, wie Zahlen mit Kommastellen umgewandelt werden.
Wir wählen hierzu den Wert $26.6875_d$, den wir zu Beginn zerlegen: $26.6875_d = 26_d + 0.6875_d$.

Zuerst wandeln wir den ganzzahligen Teil um:
\begin{align*}
    26_d \div 2 &= 13 \text{ Rest } 0 \\
    13_d \div 2 &= 6 \text{ Rest } 1 \\
    6_d \div 2 &= 3 \text{ Rest } 0 \\
    3_d \div 2 &= 1 \text{ Rest } 1 \\
    1_d \div 2 &= 0 \text{ Rest } 1
\end{align*}
Wir erhalten also: $26_d = 11010_b$

Wir wandeln nun den Bruchteil mithilfe des Horner-Schemas um:
\begin{align*}
    0.6875_d \cdot 2 &= 0.3750 + 1 \\
    0.3750_d \cdot 2 &= 0.7500 + 0 \\
    0.7500_d \cdot 2 &= 0.5000 + 1 \\
    0.5000_d \cdot 2 &= 0.0000 + 1
\end{align*}
Hier erhalten wir also: $0.6875_d = 0.1011_b$

Es folgt das Resultat: $26.6875_d = 11010.1011_b$

\subsection{Negative Zahlen}\label{subsec:negative-zahlen}

Wir haben die Zahl $-3.125_d$, die wir im 2-er System darstellen wollen.
Zu Beginn wandeln wir die Zahl ins 2-er System um: $3.125_d = 110.001_b$.
Anschliessend invertieren wir alle Bits und addieren 1 dazu:
\begin{center}
    \begin{tabular}{cccc}
        Ursprüngliche Zahl & $\dots 0 \ 0 \ 1 \ 1 \ 0. \ 0 \ 0 \ 1$ & b \\
        \hline
        1-er Komplement    & $\dots 1 \ 1 \ 0 \ 0 \ 1. \ 1 \ 1 \ 0$ & b \\
        \hline
        2-er Komplement    & $\dots 1 \ 1 \ 0 \ 0 \ 1. \ 1 \ 1 \ 1$ & b
    \end{tabular}
\end{center}
Diese Technik heisst 2-er Komplement und wird genutzt, um das Vorzeichen von Binärzahlen zu wechseln.

\subsection{Endliche Zahlen}\label{subsec:endliche-zahlen}

\begin{center}
    \begin{tabular}{|c|c|c|c|}
        \hline
        Register & Bezeichnung & \multicolumn{2}{c|}{Maximal darstellbare Zahl} \\
        \hline
        \hline
        4 Bit   & Nibble ($\frac{1}{2}$ Byte) & $0 \dots 15_d$                 & $-8 \dots \text{+}7$                                   \\
        \hline
        8 Bit   & Byte                        & $0 \dots 255_d$                & $-128 \dots \text{+}127$                               \\
        \hline
        16 Bit  & Word                        & $0 \dots 65'535_d$             & $-32'768 \dots \text{+}32'767$                         \\
        \hline
        32 Bit  & Double Word                 & $0 \dots 4.29_d \cdot 10^9$    & $-2.15 \cdot 10^9 \dots \text{+}2.15 \cdot 10^9$       \\
        \hline
        64 Bit  & Long Word                   & $0 \dots 1.84_d \cdot 10^{19}$ & $-9.22 \cdot 10^{18} \dots \text{+}9.22 \cdot 10^{18}$ \\
        \hline
        128 Bit & Double Long Word            & $0 \dots 3.40_d \cdot 10^{38}$ & $-1.70 \cdot 10^{38} \dots \text{+}1.70 \cdot 10^{38}$ \\
        \hline
    \end{tabular}
\end{center}